\documentclass[utf8]{article}

\usepackage[utf8]{inputenc}

\usepackage[parfill]{parskip}

\usepackage{amsmath}
\usepackage{amssymb}
\usepackage{amsfonts}
\usepackage{graphicx}
\usepackage{float}
\usepackage{listingsutf8}

\usepackage{fullpage}
\usepackage{hyperref}

% -----------------------------------------------------

\title{Info F-201 : Projet d’OS \\Rapport}
\author{Auteurs: Liefferinckx Romain, Rocca Manuel, Radu-Loghin Rares\\ Matricule: 000591790, 000596086, 000590079 \\ Section: INFO}
\date{10/11/2024}

\begin{document}

\maketitle
\tableofcontents

\newpage

% -----------------------------------------------------

\section{Introduction}
\subsection{Présentation du projet et contexte}
\paragraph{Dans le cadre de notre cours d'OS, nous avons réalisé un projet qui consiste à implémenter un chat en C.
Ce chat, permet la communication entre deux utilisateurs via deux terminaux différents sur un même ordinateur grâce à des 
pipes nommés pour la transmission de messages. Le chat est composé de deux parties, celle décrite ci-dessus et une autre écrite en bash,
faisant office de chat-bot. Ce chat-bot est conçu pour simuler un utilisateur en répondant automatiquement à des commandes spécifiques 
envoyées par l’interlocuteur.\\
Le projet se compose donc de deux parties : le programme de chat (chat) et le script Bash (chat-bot).}

\subsection{Objectifs du projet}
\paragraph{L'objectif de ce projet est de mettre en pratique les concepts vus en cours d'OS, notamment la gestion des processus,
des signaux, la gestion de la mémoire partagée, et la communication inter-processus. 
Ce rapport décrit les choix d'implémentation, les difficultés rencontrées et les solutions mises en œuvre utilisée dans la 
construction de ce projet.}


\section{Choix d’Implémentation}
\subsection{Choix du langage}
\paragraph{Dans le cadre de ce projet, nous avions le choix entre le C et le C++ comme langage de programmation.
Nous avons fait le chois d'utiliser du C car}
\subsection{Gestion des processus}
\paragraph{}
\subsection{Gestion des signaux}
\paragraph{Pour la gestion des processus, nous avons décidé d'utiliser "sigaction" et non pas "signal" car}
\subsection{Gestion de la mémoire partagée}
\paragraph{}
\subsection{Communication inter-processus}
\paragraph{}

\section{Difficultés Rencontrées et Solutions}
\subsection{Première difficultée: SIGINT et SIGPIPE}
\paragraph{Pour nous, la première difficultée fut celle de la gestion des signaux avec le "SIGINT" et le "SIGPIPE"}
\subsection{Deuxième difficultée}
\paragraph{}

\section{Solutions Originales et Améliorations}
\subsection{Gestion des variables globales}
\paragraph{Lors de la création du projet, étant donné que nous avons choisit le C comme langage de programmation, il était interdit 
d'utiliser de l'orienté objet et donc aucune variable dans les instanciations des objets. Le premier réflexe est donc de mettre
"const \textless nom de la variable\textgreater  = "valeur" lorsque celle-ci ne doit pas être modifié et "\textless nom de la variable\textgreater = "valeur" lorsqu'elle 
peut l'être. Nous avons alors fait le choix de mettre tout les variables globales non modifiable sous la forme 
"\#define \textless nom de la variable\textgreater \textless valeur\textgreater", cela permet }

\section{Conclusion}
\paragraph{}






\end{document}
